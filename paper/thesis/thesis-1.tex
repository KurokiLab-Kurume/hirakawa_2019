近年,畳み込みニューラルネットワーク(CNN)は,パターン・画像認識分野の様々なタスクにおいて強力な手法として知られている.
CNNを画像分類問題に適用する場合,画像とラベルデータを用いて誤差逆伝搬法によってネットワークを学習する.
これに対し,畳み込み辞書学習と呼ばれる手法が提案されている\cite{fast-csc}.
畳み込み辞書学習はフィルタと係数マップへの信号分離の最適化問題として定式化され,ラベルデータを用いずに画像のみからデータ駆動的に特徴を抽出する.
畳み込み辞書学習は最適化ベースの特徴抽出法であるため,制約や目的関数の変更が柔軟にでき,単に特徴抽出だけでなく,画像の超解像度の研究\cite{super-res}などに応用されている.

本研究では畳み込み辞書学習問題における係数マップを特徴とする,錐制約部分空間法を分類問題に適用する.
その際に,係数マップのパワースペクトルを使用することにより,データセット内の各データ間に存在する位置ずれに対する頑健性を高める.
CNNでは,新たなクラスが加わった際に,はじめから再学習する必要があるのに対して,部分空間法では,新たなクラスの張る部分空間を設計すればよいのでコストが低いことなどを考慮し,本研究では部分空間法を基にした分類器を適用する.
具体的には,パワースペクトルは非負値をとるため,特徴群は錐の形状をなすと仮定し,部分空間法に錐制約を課した錐制約部分空間法を適用する.
